
\documentclass[11pt, a4paper]{article} % Document font size and paper size
\usepackage[dvipsnames,svgnames,table]{xcolor}
\usepackage{verse} % Required for typesetting poems - this package drives this template
\usepackage{graphicx}
\usepackage{epstopdf}
\usepackage{hyperref}
\usepackage{amsmath}
\usepackage{amssymb}
\usepackage[utf8]{inputenc}
\usepackage[spanish]{babel}
\usepackage{xcolor}
\usepackage{pagecolor}
\usepackage[normalem]{ulem}

\usepackage{palatino} % Use the Palatino font by default
%\usepackage{stix} % Alternative Stix font
\setlength{\parindent}{0pt} % Disable paragraph indentation
% Author styles
\newcommand{\poemauthorcenter}[1]{\nopagebreak{\centering\footnotesize\textsc{#1}\par}} % Author as a footnote at the end of the poem center aligned
\newcommand{\poemauthorright}[1]{\nopagebreak{\raggedleft\footnotesize\textsc{#1}\par}} % Author as a footnote at the end of the poem aligned right
\renewcommand{\poemtitlefont}{\normalfont\bfseries\large\centering} % Define the poem title style
\newcommand\tab[1][1cm]{\hspace*{#1}}
\setlength{\stanzaskip}{0.75\baselineskip} % The distance between stanzas
\pagestyle{empty} % Stop page numbering through the document
%\pagecolor{yellow!30!orange}
%\pagecolor{Tan!30!white}


%%


% new LaTeX commands
\newfont{\letterbeta}{yinit.tfm}

\begin{document}
\textbf{waajacu} presenta a \textbf{SPACE MARKETING}.\\
\poemtitle{}

\settowidth{\versewidth}{DULCE DICHA OCULTA DE LOS QUE SE REBELAN SÓLOS } % Insert one of the average-sized verses, used for centering the poem

\begin{verse}[\versewidth]
{\scriptsize

Así como muestra la primera cuenta de cobro, 
el precio del producto mínimo es de 
(1'500.000 - 500-000). Ahora, además de esa implemtación 
puedo hacer un código mucho mejor, el siguiente producto máximo.\\

Desarrollarlo cuesta 36'500.000, está listo para mitad del presente año 
y funciona en cualquier plataforma Linux. \\

Así tanto, por tanto es bajo el precio en ambos casos se compra no el 
código fuente sino un ejecutable compatible con la máquina que lleve 
\textbf{una llave}. Además del costo de desarrollo en cualquier caso (máximo o mínimo), 
y para tener participación del número de dispositivos en los que se despliega 
el código cobro el costo del procesador por cada nueva llave. 
De nuevo, hace falta la llave, (que se conecta físicamente por USB a la máquina) 
para que el código funcione. Yo asumo dentro del precio de cada llave el costo de 
producir la nueva llave; con la llave les ofrezco los identificadores de cada 
llave, así pueden identificar y proteger el código que además del mío hay en 
cada hardware del negocio. \\

Si lo piensan bien, incluso además del código para contar personas también les vendo 
las llaves, sólo no vendo la llave sin vender el código. \\
Por una parte ya está terminado el produto mínimo, y su precio acordé por telefono con 
un amigo mio que trabaja con ustedes. A mi me interesa venderle el código más fino, 
y esa es la cotización de mi precio. La demostración del desempeño del código mínimo 
debe estar adjunta a esta carta. \\

No hay demostración aún del código máximo porque, y máximo no refleja el máximo; sino 
es para reflejar contrario al producto mínimo. Las especificaciones de funcionamiento 
para el código máximo son: \\

---> un tiempo de procesamiento por cada frame condicionado al hardware. ---que refleje 
buena intención en hacer que funcione rápidamente, que para las caracteristicas del 
procesador no debe exceder 6 segundos por frame para un procesador aarch64(RPI). \\

---> en la máquina una habilidad para identificar (con la intención exclusiva de conteo) 
rostros y cuerpos. Que igual que el código mínimo es deducida de la máxima posible 
versomilitud en entorno, pero que encambio la máxima de manera adaptativa 
(crecientemente más precisa) sirve al proposito de contar personas y de estimar 
un número de rostros. \\

Por ahora, les propongo usar el producto mínimo hasta que esté el producto máximo. 
El análisis demográfico, de sentimientos en los rostros, no están incluidos. \\

Rationale: para ilstrar la razón del costo, noten como aquí yo soy un obrero, 
que cobra, estimemos 40 mil pesos la hora, cubrir la meta de entregar a medio 
año requiere que use; 30 días tres meses 8 horas cada día, es así que estimo
el costo del producto. Pocas fabricas de software pueden hacer lo que hace Waajacu.\\

%------------------------------------------------
}
\end{verse}

%------------------------------------------------

\poemauthorright{www.WAAJACU.com} % Right-aligned author

%----------------------------------------------------------------------------------------

\end{document}